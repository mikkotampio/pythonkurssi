\newglossaryentry{dokumentaatio}{
	name=dokumentaatio,
	description={Ohjelmiston tai ohjelmointikielen hakuteosta muistuttava asiakirja, joka kertoo yksityiskohtaisesti sen ominaisuuksista. Vaatii yleensä esitietoja. Pythonin dokumentaatio löytyy osoitteesta \url{https://docs.python.org/3/}}
}

\newglossaryentry{komentotulkki}{
	name=komentotulkki,
	description={Interaktiivinen ohjelma, johon käyttäjä syöttää ohjelmakoodia, joka suoritetaan välittömästi. Python-komentotulkin saa auki \code{python}-komennolla; myös IDLE:ssä on komentotulkki}
}

\newglossaryentry{IDLE}{
	name=IDLE (Integrated Development and Learning Environment),
	description={Helppokäyttöinen ohjelma Python-koodin käsittelemiseen}
}

\newglossaryentry{lause}{
	name=lause,
	description={Sellainen pätkä Python-koodia, joka voi esiintyä itsenäisesti. Esimerkiksi \code{print("kissa")} tai \code{x=6} ovat lauseita}
}

\newglossaryentry{funktio}{
	name=funktio,
	description={Ohjelmoinnissa sellainen arvo, jota voidaan kutsua antamalla sille nolla tai useampi argumenttia. Funktioilla voi olla paluuarvo, sivuvaikutuksia tai ei kumpaakaan. Esimerkiksi \code{print} on funktio, joka tulostaa sille annetun argumentin}
}

\newglossaryentry{merkkijono}{
	name=merkkijono,
	description={Merkeistä koostuva pätkä tekstiä. Pythonissa merkkijonoja voi merkitä asettamalla ne lainausmerkkien sisään: esimerkiksi \code{"kissa"} on merkkijono}
}

\newglossaryentry{versiohallintaohjelma}{
	name=versiohallintaohjelma,
	description={Koodin säilyttämiseen ja historiatietojen kirjaamiseen suunniteltu ohjelmisto, joka helpottaa useamman ohjelmoijan yhteistyötä. Suosittuja ovat nykyisin mm. Git ja Mercurial}
}

\newglossaryentry{koodinvaihtomerkki}{
	name=koodinvaihtomerkki,
	description={Koodinvaihtomerkki on jokin merkki (Pythonissa kenoviiva \code{\textbackslash}), jonka avulla voidaan kirjoittaa merkkejä, jotka muuten tulkittaisiin virheellisesti. Esimerkiksi merkkijonojen sisällä lainausmerkin saa kirjoittamalla \code{\textbackslash"}, sillä pelkkä lainausmerkki tulkittaisiin merkkijonon päättymiseksi}
}

\newglossaryentry{muuttuja}{
	name=muuttuja,
	description={Lauseke, joka viittaa sille aiempin määriteltyyn arvoon. Muuttujien nimillä on joitakin rajoituksia; \code{x3} ja \code{var} ovat sallittuja, mutta \code{muut)} ja \code{4y} ovat kiellettyjä}
}

\newglossaryentry{tyyppi}{
	name=tyyppi,
	description={Jokaisella arvolla on tyyppi, joka kertoo sen ominaisuudet, kuten sen, mitä operaattoreita ja metodeja sillä on}
}

\newglossaryentry{liukuluku}{
	name=liukuluku,
	description={Pythonin vastine desimaaliluvuille. Liukuluvuilla laskeminen on niiden sisäisestä esityksestä johtuen epätarkkaa}
}
